\documentclass[]{article}
\usepackage{amsthm}
\usepackage{amsmath}
\usepackage{amsfonts}
%opening
\title{A6 Writing Exercise}
\author{Evan Curry Wilbur}

\begin{document}

\maketitle


\section*{1.}
Prove that for all natural numbers $n$, we have
$$
	\sum_{k=1}^n \frac{1}{k(k+ 1)} = \frac{n}{n + 1}
$$
\begin{proof}
	We begin the proof by induction by verifying that the base case $n = 1$ holds. Indeed it's clear to see 
	$$
		\sum_{k=1}^1 \frac{1}{k(k+ 1)} = \frac{1}{2} = \frac{1}{1 + 1}.
	$$
	With the base case established we now assume 
	$$
		\sum_{k=1}^n \frac{1}{k(k+ 1)} = \frac{n}{n+1}
	$$
	as our induction hypothesis. Then
	\begin{align*}
		\frac{n + 1}{n + 2} &= \frac{(n + 1)(n + 1)}{(n + 1)(n + 2)} \\
		&= \frac{n^2 + 2n + 1}{(n + 2)(n + 1)} \\
		&= \frac{n(n + 2) + 1}{(n + 2)(n + 1)} \\
		&= \frac{n}{n + 1} + \frac{1}{(n + 2)(n + 1)} \\
		&= \sum_{k=1}^n \frac{1}{k(k+ 1)} + \frac{1}{(n + 2)(n + 1)} \\
		&= \sum_{k=1}^{n + 1} \frac{1}{k(k+ 1)}
	\end{align*}
\end{proof}

\section*{2.}
Prove that, for all natural numbers $n$, we have that 6 divides $n^3 - n$.
\begin{proof}
	We prove the result via induction. For the base case, $n = 1$, the formula $n^3 - n = 0$ is divisible by 6. Thus the base case is satisfied. So we may take as an induction hypothesis that $6 | n^3 - n$, implying the existence of some $k \in \mathbb{N}$ such that $6k = n^3 - n$. We need to show that $6 | (n + 1)^3 - (n + 1)$. Indeed,
	\begin{align}
		(n + 1)^3 - (n + 1) &= (n^3 + 3n^2 + 3n + 1) - (n - 1) \\
		&= (n^3 - n) + (3n^2 + 3n) \\
		&= 6k + 3n(n + 1).
	\end{align}
	Notice that it must be the case that either $n$ or $n + 1$ is even. Indeed if $n$ is odd then $n + 1$ must be even. So without loss of generality let's assume that $n$ is even (as an identical argument can be made if $n + 1$ is even). Then there exists a $j \in \mathbb{N}$ such that $2j = n$. We can substitute this into (3) to give
	\begin{align*}
		6k + 3n(n + 1) &= 6k + 3(2j)(n + 1) \\
		&= 6(k + j(n + 1)).
	\end{align*}
	Hence, $6 | 6(k + j(n + 1)) = (n + 1)^3 - (n + 1)$ which proves the inductive step.
\end{proof}

\section*{3.}
Prove that the sum of the cubes of any three consecutive natural numbers is a multiple of 9.
\begin{proof}
	As if the reader and the author isn't getting tired of induction, we persist otherwise. We take as our base case $n = 1$ and see clearly that $1^3 + 2^3 + 3^3 = 36$ is divisible by 9. Hence we can assume as our induction hypothesis that $9 | n^3 + (n + 1)^3 + (n + 2)^3$, which implies the existence of a $k \in \mathbb{N}$ such that $9k = n^3 + (n + 1)^3 + (n + 2)^3$. We need to show that $9 | (n + 1)^3 + (n + 2)^3 + (n + 3)^3$. Indeed,
	\begin{align*}
		(n + 1)^3 + (n + 2)^3 + (n + 3)^3 &= n^3 + (n + 1)^3 + (n + 2)^3 + (n + 3)^3 - n^3 \\
		&= 9k + (n + 3)^3 - n^3 \\
		&= 9k + 3((n + 3)^2 + n(n+3) + n^2) \\
		&= 9k + 3(3n^2 + 9n + 9) \\
		&= 9k + 9(n^2 + 2n + 3) \\
		&= 9(k + n^2 + 2n + 3).
	\end{align*}
	Which proves the inductive step of the proof.
\end{proof}

\section*{4.}
Prove deMoivre's identity: for all natural numbers $n$, we have
$$
\left(\cos x + i\sin x\right)^n = \cos(nx) + i\sin(nx),
$$
where $i = \sqrt{-1}$. You may use these trig identities:
\begin{align*}
	\cos(A + B) &= \cos A \cos B - \sin A \sin B \\
	\sin(A + B) &= \sin A \cos B + \cos A \sin B
\end{align*}
\begin{proof}
	Euler's formula states that $e^{ix} = \cos x + i \sin x$, for every real number $x$. Thus,
	\begin{align*}
		(\cos x + i\sin x)^n &= \left(e^{ix}\right)^n \\
		&= e^{inx} \\
		&= e^{i(nx)} \\
		&= \cos(nx) + i\sin(nx)
	\end{align*}
\end{proof}
\begin{proof}
	Ok so maybe you want this proven via induction. If you're satisfied with my previous proof then ignore this.\\
	It should be clear that when $n = 1$ the identity holds. In fact it's so obvious just by looking I won't even bother writing an argument for it. So we assume for the induction hypothesis that $\left(\cos x + i\sin x\right)^n = \cos(nx) + i\sin(nx)$. We need to show that $\left(\cos x + i\sin x\right)^{n + 1} = \cos((n + 1)x) + i\sin((n + 1)x)$:
	\begin{align*}
		\cos((n + 1)x) + i\sin((n + 1)x) &= \cos(nx + x) + i\sin(nx + x) \\
		&= \cos(nx)\cos(x) - \sin(nx)\sin(x) + i(\sin(nx)\cos(x) + \cos(nx)\sin(x))\\
		&= \cos(x)\left(\cos(nx) + i\sin(nx)\right) + i\sin(x)(\cos(nx) + i\sin(nx)) \\
		&= \left(\cos(x) + i\sin(x)\right)\left(\cos(nx) + i\sin(nx)\right) \\
		&= \left(\cos(x) + i\sin(x)\right)\left(\cos(x) + i\sin(x)\right)^n \\
		&= \left(\cos(x) + i\sin(x)\right)^{n + 1}
	\end{align*}
\end{proof}

\section*{4.1.18}
	\begin{itemize}
		\item[(a)] For each natural number $n$, $1 + 4 + 7 + ... + (3n - 2) = \frac{n(3n-1)}{2}.$
		The proposition is correct but the proof is invalid. \\
		\\
			The proposition is correct but the proof is incorrect since it assumes the $k + 1$ case in the inductive step and shows that the equation does not lead to a contradiction within the equation. This, however, does not prove the inductive step. Assuming a statement and then showing that statement is true is circular.
		\begin{proof}
			The author properly establishes the base case for induction, so take that proof for our base case. For the induction hypothesis, assume
			$$
				\sum_{k=1}^n (3k - 2) = \frac{n(3n-2)}{2}.
			$$
			We need to show
			$$
				\sum_{k=1}^{n + 1} (3k - 2) = \frac{(n + 1)(3(n + 1)-2)}{2}.
			$$
			Indeed,
			\begin{align*}
				\sum_{k=1}^n (3k - 2) &= \frac{n(3n-2)}{2} \\
				(3(n + 1) - 2) + \sum_{k=1}^n (3k - 2) &= (3(n + 1) - 2) + \frac{n(3n-2)}{2} \\
				\sum_{k=1}^{n + 1} (3k - 2) &= \frac{(3n^2 - 2n) + (6n + 2)}{2} \\
				&= \frac{3n^2 + 5n + 2}{2} \\
				&= \frac{(3n + 2)(n + 1)}{2} \\
				&= \frac{(3(n + 1) - 2)(n + 1)}{2},
			\end{align*}
			which proves the induction step.
		\end{proof}
		\item[(b)] The proposition is true and the proof is correct. The only critique is that the last step of the equation doesn't \textit{exactly} match the equation, but anyone reading the proof with a sufficient level of algebra will be able to easily see the last step which is:
		$$
			\frac{(3n + 2)(n + 1)}{2} = \frac{(3(n + 1) - 2)(n + 1)}{2}.
		$$
		
	\end{itemize}
\end{document}
